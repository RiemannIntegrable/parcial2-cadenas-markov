\section{Problema 1: Optimización de Posición de Titanio en Grilla 2D}

\subsection{Introducción}

En este problema se analiza la optimización de la posición de un átomo de Titanio (Ti) en una grilla bidimensional de 4×4 que contiene átomos de Neodimio (Nd) y Hierro (Fe). El objetivo es encontrar la posición que minimiza la energía cohesiva total del sistema.

\subsubsection{Configuración del Sistema}

La grilla está compuesta por:
\begin{itemize}
    \item 4 átomos de Neodimio (Nd) ubicados en el centro formando una subgrilla 2×2
    \item 12 sitios disponibles para átomos de Fe o Ti en las posiciones restantes
    \item 1 átomo de Titanio (Ti) que debe ubicarse en uno de los 12 sitios disponibles
\end{itemize}

\subsubsection{Modelo Energético: Potencial de Morse}

La energía de interacción entre pares de átomos se calcula mediante el \textbf{Potencial de Morse}:

\begin{equation}
    U(r) = D_0 \left[ e^{-2\alpha(r-r_0)} - 2e^{-\alpha(r-r_0)} \right]
\end{equation}

donde:
\begin{itemize}
    \item $r$: distancia euclidiana entre dos átomos
    \item $D_0$: profundidad del pozo de potencial (energía de enlace)
    \item $\alpha$: parámetro que controla el ancho del pozo
    \item $r_0$: distancia de equilibrio óptima entre átomos
\end{itemize}

Los parámetros utilizados para las diferentes interacciones atómicas son:

\begin{table}[H]
\centering
\begin{tabular}{lccc}
\toprule
\textbf{Interacción} & $D_0$ & $\alpha$ & $r_0$ \\
\midrule
Ti-Nd & 0.5 & 1.5 & 2.5 \\
Ti-Fe & 0.3 & 1.2 & 2.0 \\
Nd-Nd & 1.0 & 2.0 & 2.8 \\
Fe-Fe & 0.4 & 1.3 & 2.2 \\
Nd-Fe & 0.6 & 1.6 & 2.4 \\
\bottomrule
\end{tabular}
\caption{Parámetros del Potencial de Morse para cada tipo de interacción atómica}
\label{tab:morse_params}
\end{table}

\subsection{Metodología}

Se implementaron dos enfoques para resolver el problema de optimización:

\subsubsection{Búsqueda Exhaustiva (Fuerza Bruta)}

Dado que el espacio de búsqueda es discreto y pequeño (12 posiciones posibles), se evaluaron todas las configuraciones posibles. Para cada posición $i \in \{0, 1, \ldots, 11\}$:

\begin{enumerate}
    \item Se coloca el átomo de Ti en la posición $i$
    \item Se calcula la energía total del sistema $E_i$
    \item Se registra la energía y se identifica el mínimo global
\end{enumerate}

Este método garantiza encontrar el óptimo global pero tiene complejidad $O(n)$ donde $n$ es el número de posiciones.

\begin{lstlisting}[language=Python, caption=Implementación de búsqueda exhaustiva por fuerza bruta]
def brute_force_search(grid: Grid2D) -> Dict:
    n_positions = grid.n_Fe_sites
    all_energies = np.zeros(n_positions)
    best_idx = 0
    best_energy = np.inf

    # Evaluar cada posicion posible
    for idx in range(n_positions):
        test_grid = grid.copy()
        test_grid.set_Ti_position(idx)
        energy = compute_total_energy(test_grid)
        all_energies[idx] = energy

        if energy < best_energy:
            best_energy = energy
            best_idx = idx

    return {
        'best_position_idx': best_idx,
        'best_energy': best_energy,
        'all_energies': all_energies,
        'n_evaluations': n_positions
    }
\end{lstlisting}

\subsubsection{Simulated Annealing (Recocido Simulado)}

El algoritmo de Simulated Annealing es una metaheurística inspirada en el proceso de recocido en metalurgia. Se implementó con las siguientes características:

\paragraph{Algoritmo Base}
\begin{enumerate}
    \item Inicializar en una posición aleatoria
    \item Para cada iteración $t$:
    \begin{itemize}
        \item Proponer un movimiento aleatorio a una posición vecina
        \item Calcular el cambio de energía $\Delta E = E_{\text{nueva}} - E_{\text{actual}}$
        \item Aceptar el movimiento con probabilidad:
        \begin{equation}
            P(\text{aceptar}) = \begin{cases}
                1 & \text{si } \Delta E \leq 0 \\
                e^{-\Delta E/T} & \text{si } \Delta E > 0
            \end{cases}
        \end{equation}
        \item Actualizar la temperatura según el esquema de enfriamiento
    \end{itemize}
    \item Retornar la mejor solución encontrada
\end{enumerate}

\begin{lstlisting}[language=Python, caption=Loop principal de Simulated Annealing con criterio de Metropolis-Hastings]
# Loop principal de SA
for iteration in range(1, max_iter + 1):
    # Proponer nueva posicion (uniformemente al azar)
    proposed_position = rng.integers(0, n_positions)

    # Calcular energia de la propuesta
    working_grid.set_Ti_position(proposed_position)
    proposed_energy = compute_total_energy(working_grid)

    # Cambio de energia
    delta_E = proposed_energy - current_energy

    # Temperatura actual
    T = cooling_schedule(iteration)

    # CRITERIO DE METROPOLIS-HASTINGS
    accept = False
    if delta_E <= 0:
        # Siempre aceptar mejoras
        accept = True
        n_improved += 1
    else:
        # Aceptar empeoramientos probabilisticamente
        acceptance_probability = np.exp(-delta_E / T)
        if rng.random() < acceptance_probability:
            accept = True

    # Aplicar movimiento si se acepta
    if accept:
        current_position = proposed_position
        current_energy = proposed_energy
        n_accepted += 1

        if current_energy < best_energy:
            best_position = current_position
            best_energy = current_energy
    else:
        # Rechazar: volver a posicion anterior
        working_grid.set_Ti_position(current_position)
\end{lstlisting}

\paragraph{Esquemas de Enfriamiento}

Se experimentó con diferentes esquemas de enfriamiento:

\begin{itemize}
    \item \textbf{Geométrico}: $T_{t+1} = \alpha \cdot T_t$ con $\alpha \in [0.85, 0.99]$
    \item \textbf{Exponencial}: $T_t = T_0 \cdot e^{-\beta t}$ con diferentes valores de $\beta$
\end{itemize}

\paragraph{Parámetros Utilizados}

Después de experimentación exhaustiva, se determinó que los siguientes parámetros producen convergencia óptima:

\begin{table}[H]
\centering
\begin{tabular}{ll}
\toprule
\textbf{Parámetro} & \textbf{Valor} \\
\midrule
Temperatura inicial ($T_0$) & 5.0 \\
Factor de enfriamiento ($\alpha$) & 0.98 \\
Número máximo de iteraciones & 20,000 \\
\bottomrule
\end{tabular}
\caption{Parámetros óptimos para Simulated Annealing}
\label{tab:sa_params}
\end{table}

\subsection{Resultados}

\subsubsection{Búsqueda Exhaustiva}

La búsqueda por fuerza bruta evaluó todas las 12 posiciones posibles y encontró:

\begin{itemize}
    \item \textbf{Posición óptima}: Índice 0, coordenadas $(0.0, 0.0)$
    \item \textbf{Energía mínima}: -9.744382
    \item \textbf{Número de evaluaciones}: 12
\end{itemize}

\begin{figure}[H]
\centering
\includegraphics[width=0.6\textwidth]{images/fig1_configuracion_optima.png}
\caption{Configuración óptima encontrada por búsqueda exhaustiva. El átomo de Ti (amarillo) se ubica en la posición índice 0 con coordenadas (0.0, 0.0), minimizando la energía total del sistema.}
\label{fig:configuracion_optima}
\end{figure}

\paragraph{Análisis del Paisaje Energético}

El análisis reveló que existen \textbf{dos niveles de energía} claramente diferenciados:

\begin{enumerate}
    \item \textbf{Nivel óptimo} ($E = -9.744382$): 4 posiciones
    \begin{itemize}
        \item Posiciones en las esquinas de la grilla
        \item Índices: 0, 3, 8, 11
    \end{itemize}

    \item \textbf{Nivel subóptimo} ($E = -9.157522$): 8 posiciones
    \begin{itemize}
        \item Posiciones en los bordes laterales de la grilla
        \item Índices: 1, 2, 4, 5, 6, 7, 9, 10
        \item Gap energético: $\Delta E = 0.586860$
    \end{itemize}
\end{enumerate}

Esta distribución bimodal indica una clara preferencia geométrica del sistema por las esquinas, donde el Ti se encuentra más alejado de los átomos de Nd.

\begin{figure}[H]
\centering
\includegraphics[width=0.7\textwidth]{images/fig2_distribucion_energias.png}
\caption{Distribución de energías para las 12 configuraciones posibles. Se observan claramente dos niveles de energía: 4 posiciones con energía óptima en las esquinas (-9.744) y 8 posiciones subóptimas en los bordes laterales (-9.158).}
\label{fig:distribucion_energias}
\end{figure}

\subsubsection{Resultados de Simulated Annealing}

\paragraph{Convergencia al Óptimo Global}

Con los parámetros optimizados, Simulated Annealing demostró excelente desempeño:

\begin{itemize}
    \item \textbf{Mejor energía encontrada}: -9.744382 (óptimo global)
    \item \textbf{Gap con óptimo}: $0.00000000$ (convergencia exacta)
    \item \textbf{Tasa de aceptación}: 0.193
    \item \textbf{Tasa de mejora}: 0.182
\end{itemize}

\paragraph{Robustez: Múltiples Ejecuciones}

Para validar la robustez del algoritmo, se ejecutaron 50 pruebas independientes con semillas aleatorias diferentes:

\begin{table}[H]
\centering
\begin{tabular}{ll}
\toprule
\textbf{Métrica} & \textbf{Valor} \\
\midrule
Energía promedio & $-9.744382 \pm 0.000000$ \\
Tasa de éxito (encontrar óptimo) & 100.0\% \\
Ejecuciones exitosas & 50/50 \\
Varianza de energía & $0.0000000000$ \\
\bottomrule
\end{tabular}
\caption{Resultados de 50 ejecuciones independientes de Simulated Annealing}
\label{tab:sa_multiple}
\end{table}

Estos resultados demuestran que con la configuración adecuada, Simulated Annealing alcanza consistentemente el óptimo global.

\begin{figure}[H]
\centering
\includegraphics[width=0.9\textwidth]{images/fig3_sa_evolucion.png}
\caption{Evolución de Simulated Annealing con parámetros óptimos ($T_0=5.0$, $\alpha=0.98$). Se muestra la convergencia de la energía (panel superior) y el decaimiento de la temperatura (panel inferior) a lo largo de 20,000 iteraciones.}
\label{fig:sa_evolucion}
\end{figure}

\paragraph{Análisis de Sensibilidad de Parámetros}

Se realizó un grid search exhaustivo probando 4 valores de temperatura inicial ($T_0 \in \{1.0, 2.0, 5.0, 10.0\}$) y 7 valores del factor de enfriamiento ($\alpha \in \{0.85, 0.90, 0.92, 0.95, 0.97, 0.98, 0.99\}$), para un total de 28 combinaciones:

\begin{itemize}
    \item \textbf{Configuraciones que encontraron el óptimo}: 28/28
    \item \textbf{Tasa de éxito global}: 100.0\%
\end{itemize}

Este resultado demuestra que el algoritmo es \textbf{robusto} ante variaciones en los parámetros, posiblemente debido a:
\begin{enumerate}
    \item El espacio de búsqueda discreto y pequeño (12 posiciones)
    \item La existencia de múltiples óptimos globales (4 posiciones con la misma energía mínima)
    \item El número suficiente de iteraciones (10,000-20,000) que permite exploración exhaustiva
\end{enumerate}

\begin{figure}[H]
\centering
\includegraphics[width=0.85\textwidth]{images/fig5_comparacion_esquemas.png}
\caption{Comparación de diferentes esquemas de enfriamiento (geométrico y exponencial) mostrando la energía final alcanzada y la tasa de aceptación. Todos los esquemas probados convergen al óptimo global.}
\label{fig:comparacion_esquemas}
\end{figure}

\begin{figure}[H]
\centering
\includegraphics[width=\textwidth]{images/fig6_grid_search_parametros.png}
\caption{Análisis exhaustivo del grid search de parámetros ($T_0 \times \alpha$). Los heatmaps muestran la tasa de éxito y la tasa de aceptación para las 28 combinaciones probadas. Todos los parámetros encuentran el óptimo global (100\% de éxito).}
\label{fig:grid_search}
\end{figure}

\subsection{Análisis Físico: Hipótesis de Skelland}

\subsubsection{Hipótesis}

Según Skelland, se espera que el Titanio prefiera sitios \textbf{alejados de las tierras raras} (Neodimio), es decir, que la posición óptima maximice la distancia promedio Ti-Nd.

\subsubsection{Análisis de Contribuciones Energéticas}

Para la posición óptima (índice 0, coordenadas $(0.0, 0.0)$):

\begin{table}[H]
\centering
\begin{tabular}{ll}
\toprule
\textbf{Componente} & \textbf{Valor} \\
\midrule
Energía Ti-Nd & -0.390298 \\
Energía Ti-Fe & -1.647790 \\
Energía total de Ti & -2.038087 \\
\midrule
Distancia mínima a Nd & 3.9598 Å \\
Distancia promedio a Nd & 6.1003 Å \\
Distancia mínima a Fe & 2.8000 Å \\
\bottomrule
\end{tabular}
\caption{Análisis energético y geométrico de la posición óptima}
\label{tab:optimal_analysis}
\end{table}

\subsubsection{Comparación: Mejor vs. Peor Posición}

\begin{table}[H]
\centering
\begin{tabular}{lcc}
\toprule
\textbf{Métrica} & \textbf{Mejor (índice 0)} & \textbf{Peor (índice 1)} \\
\midrule
Coordenadas & $(0.0, 0.0)$ & $(0.0, 2.8)$ \\
Energía total & -9.744382 & -9.157522 \\
Dist. promedio a Nd (Å) & 6.1003 & 4.6552 \\
Dist. mínima a Nd (Å) & 3.9598 & 2.8000 \\
\bottomrule
\end{tabular}
\caption{Comparación entre la mejor y peor configuración}
\label{tab:best_vs_worst}
\end{table}

\subsubsection{Correlación Distancia-Energía}

Se analizó la correlación entre la distancia promedio Ti-Nd y la energía total para las 12 posiciones:

\begin{itemize}
    \item \textbf{Correlación de Pearson}: $r = -0.9196$
    \item \textbf{P-value}: $1.3477 \times 10^{-5}$ (altamente significativo)
\end{itemize}

\begin{figure}[H]
\centering
\includegraphics[width=0.8\textwidth]{images/fig4_correlacion_distancia_energia.png}
\caption{Correlación entre distancia promedio Ti-Nd y energía total. La correlación negativa significativa ($r=-0.9196$, $p<0.001$) indica que a mayor distancia al Nd, menor es la energía (configuración más estable), confirmando la hipótesis de Skelland.}
\label{fig:correlacion}
\end{figure}

Dado que $r < 0$ con significancia estadística ($p < 0.001$), se observa una \textbf{correlación negativa significativa}:

\begin{center}
\textit{A MAYOR distancia promedio a Nd, MENOR energía (más estable)}
\end{center}

\subsubsection{Conclusión sobre la Hipótesis de Skelland}

\begin{tcolorbox}[colback=green!5!white, colframe=green!75!black, title=Resultado]
\textbf{La hipótesis de Skelland es CONFIRMADA} para este sistema.

\vspace{0.5em}

Los datos experimentales muestran que:
\begin{enumerate}
    \item La posición óptima tiene \textbf{MAYOR} distancia promedio a Nd (6.1003 Å) que la peor posición (4.6552 Å)
    \item Existe correlación negativa significativa entre distancia a Nd y energía ($r = -0.9196$, $p < 0.001$)
    \item El Ti prefiere posiciones \textbf{MÁS ALEJADAS} de los átomos de Nd (esquinas de la grilla)
\end{enumerate}

\vspace{0.5em}

\textbf{Interpretación Física:} Las interacciones Ti-Nd, aunque atractivas (contribución energética negativa), son menos favorables que las interacciones Ti-Fe. El sistema minimiza su energía colocando el Ti en las esquinas de la grilla, maximizando la distancia promedio a los átomos de Nd mientras mantiene interacciones favorables con los átomos de Fe. Las posiciones en los bordes laterales, más cercanas a Nd, resultan en mayor energía total.
\end{tcolorbox}

\begin{figure}[H]
\centering
\includegraphics[width=\textwidth]{images/fig7_mejor_vs_peor.png}
\caption{Comparación visual entre la mejor configuración (izquierda, índice 0) y la peor configuración (derecha, índice 1). La posición óptima se encuentra en una esquina (máxima distancia a Nd) mientras que la peor está en un borde lateral (más cercana a Nd).}
\label{fig:mejor_vs_peor}
\end{figure}

\subsection{Conclusiones}

\begin{enumerate}
    \item Se identificó exitosamente la posición óptima para el átomo de Ti mediante búsqueda exhaustiva: posición $(0.0, 0.0)$ con energía mínima de $-9.744382$

    \item Existen 4 posiciones equivalentes con energía óptima ubicadas en las esquinas de la grilla (degeneración debido a simetría del sistema)

    \item Simulated Annealing demostró convergencia robusta al óptimo global con tasa de éxito del 100\% en 50 ejecuciones independientes

    \item El algoritmo es robusto ante variaciones de parámetros: todas las 28 combinaciones probadas encontraron el óptimo

    \item La hipótesis de Skelland fue confirmada: el Ti prefiere posiciones más alejadas de los átomos de Nd, ubicándose en las esquinas de la grilla

    \item El análisis de correlación reveló una relación negativa significativa ($r = -0.9196$, $p < 0.001$) entre distancia a Nd y energía, indicando que a mayor distancia, menor energía (configuración más estable)
\end{enumerate}
