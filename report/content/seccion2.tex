\section{Problema 2: Desafío Combinatorio en 2D}

\subsection{Introducción}

En el Problema 2 se incrementa significativamente la complejidad del sistema. Ahora trabajamos con una grilla 2D de dimensiones $10 \times 10$, que contiene 84 átomos de Hierro (Fe), 16 átomos de Neodimio (Nd) ubicados en un cuadrado central en las posiciones $(x,y)$ con $x, y \in \{3, 4, 5, 6\}$, y 8 átomos de Titanio (Ti) a optimizar.

El objetivo es encontrar la configuración óptima para colocar los 8 átomos de Titanio en los sitios de Fe que minimice la energía cohesiva total del sistema. El número de configuraciones posibles es $\binom{84}{8} \approx 1.5 \times 10^{10}$, lo que hace inviable el uso de fuerza bruta como en el Problema 1.

\begin{figure}[H]
\centering
\includegraphics[width=0.7\textwidth]{images/punto2_configuración_inicial.png}
\caption{Configuración inicial aleatoria del sistema 10×10 con energía 51023.177. Cuadrado amarillo: 16 átomos de Nd fijos. Círculos naranjas: 8 átomos de Ti distribuidos aleatoriamente.}
\label{fig:config_inicial_p2}
\end{figure}

\subsection{Punto 1: Definición de Estado y Movimientos}

\subsubsection{Espacio de Estados}

Un estado del sistema está completamente caracterizado por el conjunto de 8 coordenadas $(x_i, y_i)$ que especifican las posiciones de los átomos de Titanio en la grilla. Formalmente:

\begin{equation}
S = \{(x_1, y_1), (x_2, y_2), \ldots, (x_8, y_8)\} \quad \text{con} \quad x_i, y_i \in \{0, 1, \ldots, 9\}
\end{equation}

donde cada posición corresponde a un sitio previamente ocupado por un átomo de Fe.

\subsubsection{Movimiento en la Cadena de Markov}

Para explorar el espacio de estados, se implementó un movimiento tipo \textit{swap} (intercambio) que selecciona aleatoriamente un átomo de Ti de la configuración actual, selecciona aleatoriamente un átomo de Fe, e intercambia sus posiciones. Este movimiento garantiza ergodicidad: es posible alcanzar cualquier configuración válida desde cualquier otra mediante una secuencia finita de intercambios.

\subsection{Punto 2: Implementación de Simulated Annealing}

\subsubsection{Parámetros del Algoritmo}

Los parámetros seleccionados para el algoritmo de Simulated Annealing fueron: temperatura inicial $T_0 = 1000$, tasa de enfriamiento $\alpha = 0.995$, y número máximo de iteraciones $\text{max\_iter} = 3000$.

\subsubsection{Justificación del Esquema de Enfriamiento}

El esquema de enfriamiento geométrico utilizado es:

\begin{equation}
T_n = T_0 \cdot \alpha^n
\end{equation}

donde $n$ es el número de iteración. Con $\alpha = 0.995$, se obtiene un enfriamiento gradual que permite exploración inicial (alta $T$) para aceptar movimientos desfavorables y escapar de mínimos locales, seguida de convergencia final (baja $T$) para refinamiento hacia el óptimo global. La temperatura inicial $T_0 = 1000$ fue seleccionada para ser significativamente mayor que las variaciones típicas de energía $\Delta E$ en el sistema, garantizando alta probabilidad de aceptación inicial.

\subsubsection{Cálculo Incremental de $\Delta E$}

La eficiencia computacional del algoritmo se logró mediante el cálculo incremental de la diferencia de energía. Dado que un intercambio solo afecta las interacciones locales de los dos átomos involucrados, se calculó:

\begin{equation}
\Delta E = E_{\text{nuevo}} - E_{\text{actual}} = \sum_{\text{vecinos}} \left[ U_{\text{nuevo}}(r) - U_{\text{actual}}(r) \right]
\end{equation}

sin necesidad de recalcular la energía total de toda la grilla, reduciendo la complejidad de $O(N^2)$ a $O(N)$ por iteración.

\subsubsection{Ejecución Masiva en Paralelo}

Para garantizar la robustez de los resultados, se ejecutaron \textbf{100,000 runs independientes} del algoritmo en paralelo, aprovechando 16 cores del procesador. El tiempo total de ejecución fue de 93.25 segundos ($\sim$0.001 s/run).

\subsection{Punto 3: Análisis de Resultados}

\subsubsection{Estadísticas de Convergencia}

Los resultados de los 100,000 runs mostraron:

\begin{table}[h]
\centering
\begin{tabular}{lr}
\hline
\textbf{Métrica} & \textbf{Valor} \\
\hline
Mejor energía encontrada & 48679.973344 \\
Energía media & $48681.717 \pm 8.953$ \\
Peor energía & 48728.333086 \\
Mejora promedio & 4.40\% \\
\hline
\end{tabular}
\caption{Estadísticas energéticas de 100,000 runs de Simulated Annealing}
\end{table}

La baja desviación estándar ($\sigma = 8.95$) indica que el algoritmo converge consistentemente a configuraciones de energía similar, sugiriendo que el óptimo global fue alcanzado.

\subsubsection{Análisis de la Convergencia Temporal}

El mejor run (ID: 27558) alcanzó el óptimo en la iteración 1614 de 3000 (53.8\% del tiempo total), con temperatura $T = 0.307$. Las estadísticas generales de convergencia muestran una media de 1473 iteraciones, mediana de 1365 iteraciones, mínimo de 248 iteraciones (run más rápido), y máximo de 2999 iteraciones (run más lento). Esto indica que el algoritmo está \textbf{bien calibrado}, con un balance óptimo entre exploración y convergencia, ya que la mayoría de los runs convergen antes del 50\% del tiempo asignado.

\subsubsection{Configuración Óptima}

La configuración de mínima energía se muestra en la Figura~\ref{fig:config_optima_p2}. Visualmente, se observa que los átomos de Ti (naranja con borde punteado) tienden a ubicarse alrededor del cuadrado central de átomos de Nd (amarillo), formando un patrón simétrico en pares. Esta configuración logra una energía de 48679.973, representando una mejora del 4.6\% respecto a la configuración inicial.

\begin{figure}[H]
\centering
\includegraphics[width=0.7\textwidth]{images/punto2_configuracion_optima.png}
\caption{Configuración óptima (Run 27558, energía 48679.973). Los 8 átomos de Ti (naranja con borde punteado) forman pares simétricos alrededor del cuadrado central de Nd (amarillo).}
\label{fig:config_optima_p2}
\end{figure}

\subsubsection{Análisis del Patrón Espacial}

Se realizó un análisis cuantitativo de las características espaciales de la configuración óptima:

\begin{table}[h]
\centering
\begin{tabular}{lr}
\hline
\textbf{Métrica Espacial} & \textbf{Valor} \\
\hline
Distancia promedio Ti-Nd & 1.000 Å \\
Distancia promedio Ti-Ti & 3.626 Å \\
Clustering score & 0.715 \\
Distancia al centro promedio & 2.550 \\
\hline
\end{tabular}
\caption{Métricas espaciales de la configuración óptima}
\end{table}

\begin{figure}[H]
\centering
\includegraphics[width=\textwidth]{images/punto2_metricas.png}
\caption{Análisis espacial de la configuración óptima. Distancias Ti-Nd (promedio 1.0 Å), distancias Ti-Ti (promedio 3.626 Å), y clustering score de 0.715 indican agrupamiento moderado cerca del centro.}
\label{fig:metricas_p2}
\end{figure}

\subsubsection{Interpretación Física}

El análisis revela que los átomos de Ti se ubican relativamente cerca de los átomos de Nd (distancia promedio: 1.00 Å), lo que contrasta parcialmente con la hipótesis del Problema 1 que sugería que el Ti prefiere sitios alejados de las tierras raras. Los Ti tienden a agruparse (clustering score: 0.715) con una distancia promedio Ti-Ti de 3.63 Å, sugiriendo que las interacciones Ti-Ti favorables juegan un rol importante. La ubicación cerca del centro de la grilla (distancia promedio: 2.55) corresponde a posiciones intermedias dado el tamaño 10×10.

El patrón NO confirma completamente la hipótesis del Punto 1. La minimización de energía resulta de un balance complejo entre interacciones Ti-Nd (potencialmente repulsivas a corta distancia), Ti-Ti (favorables según parámetros del potencial de Morse), y Ti-Fe (que conforman la mayoría del sistema).

\subsubsection{Fases de Exploración y Explotación}

La gráfica de evolución energética (Figura~\ref{fig:energia_vs_iter_p2}) muestra las dos fases características del Simulated Annealing. La fase de exploración (iteraciones 0-241, $T > 0.3 T_0$, región naranja) presenta alta variabilidad energética con fluctuaciones significativas, aceptando movimientos desfavorables frecuentemente. La fase de explotación (iteraciones 241-3000, $T \leq 0.3 T_0$, región verde) muestra convergencia gradual hacia el óptimo con fluctuaciones decrecientes. El punto de convergencia al óptimo (iteración 1614, línea púrpura) se encuentra en la zona de explotación, evidenciando el balance adecuado del esquema de enfriamiento.

\begin{figure}[H]
\centering
\includegraphics[width=\textwidth]{images/punto2_evolucionenergia.png}
\caption{Evolución de energía durante SA. Superior: energía actual (azul) y mejor energía (magenta). Regiones sombreadas: exploración (naranja) y explotación (verde). Línea púrpura: convergencia al óptimo (iter 1614). Inferior: temperatura en escala log.}
\label{fig:energia_vs_iter_p2}
\end{figure}

\subsubsection{Validación Estadística con 100,000 Runs}

Para validar la robustez del algoritmo, se ejecutaron 100,000 runs independientes en paralelo. La Figura~\ref{fig:resultados_p2} presenta un análisis comprehensivo de estos resultados:

\begin{figure}[H]
\centering
\includegraphics[width=\textwidth]{images/punto2_resultados.png}
\caption{Análisis de 100,000 runs. Superior izquierda: evolución top 5 runs. Superior derecha: distribución de energías finales ($\sigma=8.95$). Inferior: mejora relativa por run (4.4-5.6\%). Mejor run: 48679.973.}
\label{fig:resultados_p2}
\end{figure}

La distribución extremadamente estrecha de energías finales (99.99\% de los runs convergen a energías entre 48680-48690) proporciona fuerte evidencia de que el algoritmo efectivamente encuentra el óptimo global, no mínimos locales. La iteración promedio de convergencia (1473) confirma que el esquema de enfriamiento está bien calibrado.

\subsection{Conclusiones del Problema 2}

El algoritmo de Simulated Annealing con los parámetros seleccionados ($T_0=1000$, $\alpha=0.995$, max\_iter=3000) demostró ser robusto y eficiente para este problema combinatorio de $\sim 10^{10}$ configuraciones. La implementación del cálculo incremental de $\Delta E$ permitió ejecutar 100,000 runs en tiempo razonable (~93 segundos), validando estadísticamente la convergencia al óptimo global con desviación estándar extremadamente baja ($\sigma = 8.95$).

El patrón espacial óptimo reveló un comportamiento más complejo que la hipótesis inicial: los átomos de Ti no simplemente evitan a los Nd, sino que buscan un balance energético considerando todas las interacciones atómicas del sistema, distribuyéndose en pares simétricos alrededor del cuadrado central de Nd. El análisis de las fases de exploración/explotación confirma que el esquema de enfriamiento está bien calibrado, con transición a explotación en la iteración 241 (8\% del tiempo) y convergencia típica alrededor del 50\% del tiempo total asignado.
